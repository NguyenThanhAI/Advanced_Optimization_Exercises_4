\documentclass[14pt, a4paper]{article}
\usepackage{minitoc}
\usepackage[left=3.00cm, right=2.5cm, top=2.00cm, bottom=2.00cm]{geometry}
\usepackage{amsmath}
\usepackage{amssymb}
\usepackage{amsthm}
\usepackage{mathtools}
\usepackage{graphicx}
\usepackage{algpseudocode}
\usepackage{algorithm}
\usepackage{blindtext}
\usepackage{setspace}
\usepackage[utf8]{inputenc}
\usepackage[utf8]{vietnam}
\usepackage[center]{caption}
\usepackage[shortlabels]{enumitem}
\usepackage{fancyhdr} % header, footer
\usepackage{hyperref} % loại bỏ border với mục lục và công thức
\usepackage[nonumberlist, nopostdot, nogroupskip]{glossaries}
\usepackage{glossary-superragged}
\setglossarystyle{superraggedheaderborder}
\pagestyle{fancy}
%\usepackage[style=numeric,sortcites]{biblatex}
%\addbibresource{ref.bib}
%\usepackage[numbers]{natbib}
\usepackage{indentfirst}
\usepackage[natbib,backend=biber,style=ieee, sorting=ynt]{biblatex}
\bibliography{ref.bib}

\graphicspath{{./figures/}}

%\renewbibmacro*{cite}{%
%  \printtext[bibhyperref]{%
%    \printfield{prefixnumber}%
%    \printfield{labelnumber}%
%    \ifbool{bbx:subentry}%
%      {\printfield{entrysetcount}}%
%   \ifnumequal{\value{citecount}}{\value{citetotal}-1}%
%       {\gdef\multicitedelim{\addspace\bibstring{and}\space}}%
%       {\gdef\multicitedelim{\addcomma\space}}%
%    }%
%}

%\makenoidxglossaries
%
%% Danh mục thuật ngữ
%\newglossaryentry{GMRES}
%{
%	name={GMRES},
%	description={Generalized Minimal Residual Method}
%}
%
%\newglossaryentry{MINRES}
%{
%	name={MINRES},
%	description={Minimal Residual Method}
%}
%
%\hypersetup{
%    colorlinks=false,
%    pdfborder={0 0 0},
%}
%
%\title{Tiểu luận phương pháp số cho đại số tuyến tính}
%
%\author{Nguyễn Chí Thanh}
%
%%\date{24-04-2022}
\fancyhf{}
\rhead{\textbf{Môn học: Tối ưu hóa nâng cao}}
\lhead{\textbf{GVHD: TS. Hoàng Nam Dũng}}
\rfoot{\thepage}
\lfoot{\textbf{Học viên thực hiện: Nguyễn Chí Thanh - Nguyễn Đức Thịnh}}
\renewcommand{\headrulewidth}{0.4pt}
\renewcommand{\footrulewidth}{0.4pt}
%
%\numberwithin{equation}{section}
%\numberwithin{algorithm}{section}
%\numberwithin{figure}{section}
%
%\setlength{\parindent}{0.5cm}
%
%\setcounter{secnumdepth}{3} % Cho phép subsubsection trong report
%\setcounter{tocdepth}{3} % Chèn subsubsection vào bảng mục lục

%\newtheorem{dl}{Định lý}
%\newtheorem{md}{Mệnh đề}
%\newtheorem{bd}{Bổ đề}
%\newtheorem{dn}{Định nghĩa}
%\newtheorem{hq}{Hệ quả}

%\newtheorem{baitap}{Bài tập}
%\newtheorem*{loigiai}{Lời giải}

%\numberwithin{dl}{section}
%\numberwithin{md}{section}
%\numberwithin{bd}{section}
%\numberwithin{dn}{section}
%\numberwithin{hq}{section}

\setlength{\parindent}{0cm}

\newtheorem{dl}{Định lý}
\newtheoremstyle{sltheorem}
{}                % Space above
{}                % Space below
{\normalfont}        % Theorem body font % (default is "\upshape")
{}                % Indent amount
{\bfseries}       % Theorem head font % (default is \mdseries)
{.}               % Punctuation after theorem head % default: no punctuation
{ }               % Space after theorem head
{}                % Theorem head spec
\theoremstyle{sltheorem}
\newtheorem{baitap}{Bài tập}
\newtheoremstyle{soltheorem}
{}                % Space above
{}                % Space below
{\normalfont}        % Theorem body font % (default is "\upshape")
{}                % Indent amount
{\bfseries}       % Theorem head font % (default is \mdseries)
{.}               % Punctuation after theorem head % default: no punctuation
{\newline}               % Space after theorem head
{}                % Theorem head spec
\theoremstyle{soltheorem}
\newtheorem*{loigiai}{Lời giải}

\onehalfspacing

\begin{document}

    \begin{titlepage}

        \newcommand{\HRule}{\rule{\linewidth}{0.5mm}} % Defines a new command for the horizontal lines, change thickness here

        \center % Center everything on the page

        %----------------------------------------------------------------------------------------
        %	HEADING SECTIONS
        %----------------------------------------------------------------------------------------
        \textsc{\LARGE Đại học Quốc Gia Hà Nội}\\[0.5cm]
        \textsc{\LARGE Trường đại học Khoa học tự nhiên}\\[0.5cm] % Name of your university/college
        \textsc{\LARGE Khoa Toán - Cơ - Tin học}\\[0.5cm]

        \includegraphics[scale=0.2]{HUS-logo.jpg}\\[0.5cm]

        \textsc{\Large Chuyên ngành: Khoa học dữ liệu}\\[0.5cm] % Major heading such as course name


        %----------------------------------------------------------------------------------------
        %	TITLE SECTION
        %----------------------------------------------------------------------------------------

        \HRule \\[0.4cm]
        { \huge \bfseries Bài tập môn học}\\[0.4cm] % Title of your document
        \HRule \\[1.5cm]

        \textsc{\Large Môn học: Tối ưu hóa nâng cao}\\[1.5cm] % Minor heading such as course title


        \textsc{\Large Bài tập 4}\\[1.5cm]


        %----------------------------------------------------------------------------------------
        %	AUTHOR SECTION
        %----------------------------------------------------------------------------------------
        \begin{minipage}{0.4\textwidth}
            \begin{flushleft} \Large
            \emph{Giảng viên hướng dẫn:} \\
            TS. Hoàng Nam Dũng % Supervisor's Name
            \end{flushleft}
        \end{minipage}\\[1cm]

        \begin{minipage}{0.4\textwidth}
        \begin{flushleft} \Large
        \emph{Nhóm học viên thực hiện:}\\
        Nguyễn Chí Thanh \\
        MSHV: 21007925 \\ % Your name
        Nguyễn Đức Thịnh \\
        MSHV: 21007923 \\
        Lớp: Khoa học dữ liệu - K4
        \end{flushleft}
        \end{minipage}


        % If you don't want a supervisor, uncomment the two lines below and remove the section above
        %\Large \emph{Author:}\\
        %John \textsc{Smith}\\[3cm] % Your name

        %----------------------------------------------------------------------------------------
        %	DATE SECTION
        %----------------------------------------------------------------------------------------

        % I don't want day because it is English
        % {\large \today}\\[2cm] % Date, change the \today to a set date if you want to be precise

        %----------------------------------------------------------------------------------------
        %	LOGO SECTION
        %----------------------------------------------------------------------------------------

        %\includegraphics{logo/rsz_3logo-khtn.png}\\[1cm] % Include a department/university logo - this will require the graphicx package

        %----------------------------------------------------------------------------------------

        \vfill % Fill the rest of the page with whitespace

    \end{titlepage}

    \nocite{*}

    \newpage

    \begin{baitap}
        Cho $A \in \mathbb{R}^{m\times n}$ và $rank(A)=m$. Xét bài toán tối ưu hóa
        \begin{equation*}
            \begin{aligned}
                &\min \dfrac{1}{2} \lVert x \rVert_2^2 \\ &\text{s.t. }Ax=y
            \end{aligned}
        \end{equation*}
        \begin{enumerate}[wide, labelwidth=!, labelindent=0pt,label=\textbf{\arabic*}.]
            \item Hãy chứng minh rằng bài toán tối ưu trên là một bài toán lồi.
            \item Hãy xây dựng hàm Lagrange của bài toán.
            \item Hãy trình bày điều kiện KKT của bài toán.
            \item Từ điều kiện KKT hãy tìm nghiệm tối ưu của bài toán.
        \end{enumerate}
    \end{baitap}

    \begin{loigiai}

        Cho $A \in \mathbb{R}^{m\times n}$ và $rank(A)=m$. Xét bài toán tối ưu hóa
        \begin{equation*}
            \begin{aligned}
                &\min \dfrac{1}{2} \lVert x \rVert_2^2 \\ &\text{s.t. }Ax=y
            \end{aligned}
        \end{equation*}
        
        \begin{enumerate} [wide, labelwidth=!, labelindent=0pt,label=\textbf{\arabic*}.]
            \item Hãy chứng minh rằng bài toán tối ưu trên là một bài toán lồi.
            
            Ta chứng minh $f(x)=\dfrac{1}{2} \lVert x \rVert_2^2$ là một hàm lồi.
            
            Ta có $dom(f)=\mathbb{R}^n$ là tập lồi.
            \begin{equation*}
                f(x)=\dfrac{1}{2} \lVert x \rVert_2^2=\dfrac{1}{2}x^T x=\dfrac{1}{2}x^T I x
            \end{equation*}

            Ta tính gradient của $f(x)$ theo $x$:

            \begin{equation*}
                \nabla f(x)=\dfrac{1}{2} \Big\lbrack I + I^T \Big\rbrack x=\dfrac{1}{2}2Ix=x
            \end{equation*}

            Ta tính ma trận Hessian của $f(x)$ theo $x$:
            \begin{equation*}
                \nabla^2 f(x)=I
            \end{equation*}

            Ta xét:

            \begin{equation*}
                p^T \nabla^2 f(x) p=p^T I p = p^T p \geq 0 \forall p \in \mathbb{R}^n
            \end{equation*}

            nên:

            \begin{equation*}
                \nabla^2 f(x) \succeq 0 \forall x \in \mathbb{R}^n
            \end{equation*}

            Vì $dom(f)$ là tập lồi và $\nabla^2 f(x) \succeq 0$ nên $f(x)=\dfrac{1}{2} \lVert x \rVert_2^2$ là một hàm lồi.
            Bài toán tối ưu hóa trên có $f(x)$ là hàm lồi, không có ràng buộc bất đẳng thức và ràng buộc đẳng thức $Ax=y$ là hàm affine.
            Vì vậy bài toán tối ưu trên là một bài toán lồi.

            \item Hãy xây dựng hàm Lagrange của bài toán.
            
            Ta xét bài toán tối ưu dạng tổng quát:

            \begin{equation*}
                \begin{aligned}
                    & \min_x f(x) \\
                     \text{s.t } & h_i(x) \leq 0, i = 1, \dots, k \\
                    & l_j(x) = 0, j = 1, \dots, l.
                \end{aligned}
            \end{equation*}
            Hàm Lagrange của bài toán tối ưu được định nghĩa là:

            \begin{equation*}
                L(x, u, v) = f(x) + \sum_{i=1}^k u_i h_i(x) + \sum_{j=1}^l v_j l_j (x)
            \end{equation*}
            với các biến mới $u \in \mathbb{R}^k$ và $v \in \mathbb{R}^l$ và ràng buộc $u \geq 0$.

            Đối với bài toán tối ưu ở đề bài chỉ có ràng buộc đẳng thức mà không có ràng buộc bất đẳng thức nên hàm Lagrange của bài toán tối ưu ở đề bài là:

            \begin{equation*}
                L(x, v)=f(x) + v^T\Big \lbrack Ax - y \Big \rbrack=\dfrac{1}{2}\lVert x \rVert_2^2 + v^T \Big \lbrack Ax - y \Big \rbrack \text{ (do không có ràng buộc bất đẳng thức) } v \in \mathbb{R}^m
            \end{equation*}

            \item Hãy trình bày điều kiện KKT của bài toán.
            
            Ta xét bài toán tối ưu dạng tổng quát:

            \begin{equation*}
                \begin{aligned}
                    & \min_x f(x) \\
                     \text{s.t } & h_i(x) \leq 0, i = 1, \dots, k \\
                    & l_j(x) = 0, j = 1, \dots, l.
                \end{aligned}
            \end{equation*}

            Điều kiện Karush - Kuhn - Tucker (KKT) của bài toán tối ưu này là:

            \begin{equation*}
                \begin{cases}
                    0 \in \partial \Bigg( f(x) + \displaystyle\sum_{i=1}^k u_i h_i(x) + \displaystyle\sum_{j=1}^l v_j l_j(x) \Bigg) \\
                    u_i h_i(x) = 0 \thickspace \forall \thickspace i = 1, \dots, k \\
                    h_i(x) \leq 0, l_j(x) = 0 \thickspace \forall \thickspace i=1, \dots, k, j=1, \dots, l \\
                    u_i \geq 0 \thickspace \forall i = 1, \dots, k
                \end{cases}
            \end{equation*}

            Đối với bài toán tối ưu ở đề bài chỉ có ràng buộc đẳng thức mà không có ràng buộc bất đẳng thức nên điều kiện KKT của bài toán tối ưu ở đề bài là:

            \begin{equation*}
                \begin{aligned}
                &\begin{cases}
                    0 \in \partial \Big ( f(x) + v^T\Big \lbrack Ax - y \Big \rbrack \Big) \\ Ax - y = 0
                \end{cases}\\
                \Leftrightarrow & \begin{cases} 0 \in \partial \Big( \dfrac{1}{2} \lVert x \rVert_2^2 + v^T\Big \lbrack Ax - y \Big \rbrack \Big) \\ Ax - y = 0 \end{cases}
                \end{aligned}
            \end{equation*}

            Ta chứng minh $f(x) + v^T\Big \lbrack Ax - y \Big \rbrack$ là hàm lồi.

            Ta xét hàm số:

            \begin{equation*}
                \begin{aligned}
                    g(x) &= f(x) + v^T\Big \lbrack Ax - y \Big \rbrack \\
                    &=\dfrac{1}{2} \lVert x \rVert_2^2 + v^T\Big \lbrack Ax - y \Big \rbrack
                \end{aligned}
            \end{equation*}

            Ta có $dom(g)= \mathbb{R}^n$ là tập lồi.

            \begin{equation*}
                \begin{aligned}
                    g(x) &= \dfrac{1}{2} \lVert x \rVert_2^2 + v^T\Big \lbrack Ax - y \Big \rbrack \\
                    &=\dfrac{1}{2}x^T x + v^T\Big \lbrack Ax - y \Big \rbrack \\
                    &= \dfrac{1}{2}x^T I x + v^T\Big \lbrack Ax - y \Big \rbrack
                \end{aligned}
            \end{equation*}
            Ta tính gradient của $g(x)$ theo $x$:
            \begin{equation*}
                \nabla g(x) = \dfrac{1}{2} \Big \lbrack I + I^T \Big \rbrack x + A^T v = \dfrac{1}{2}2I x + A^T x = x + A^T v
            \end{equation*}
            Ta tính ma trận Hessian của $g(x)$ theo $x$:

            \begin{equation*}
                \nabla^2 g(x) = I
            \end{equation*}

            Ta xét:

            \begin{equation*}
                p^T \nabla^2 g(x) p = p^T I p = p^T p \geq 0 \thickspace \forall \thickspace p \in \mathbb{R}^n
            \end{equation*}

            nên:

            \begin{equation*}
                \nabla^2 g(x) \succeq 0 \thickspace \forall \thickspace x \in \mathbb{R}^n
            \end{equation*}

            Vì $dom(g)=\mathbb{R}^n$ và $\nabla^2 g(x) \succeq 0$ nên $g(x) = \dfrac{1}{2} \lVert x \rVert_2^2 + v^T\Big \lbrack Ax - y \Big \rbrack$ là hàm lồi.

            Ta nhận thấy $g(x)$ là một hàm khả vi với mọi $x \in \mathbb{R}^n$ và $g(x)$ là một hàm lồi nên ta có thể tính dưới vi phân $\partial g(x)$:

            \begin{equation*}
                \begin{aligned}
                    \partial g(x) &= \lbrace \nabla g(x) \rbrace \\
                    &= \lbrace x + A^T v \rbrace
                \end{aligned}
            \end{equation*}

            Để $0 \in \partial g(x) \Rightarrow \nabla g(x) = 0 \Rightarrow x + A^T v = 0$.
            Ta viết lại điều kiện KKT:

            \begin{equation*}
                \begin{cases} x + A^Tv=0 \\ Ax - y = 0 \end{cases}
            \end{equation*}

            \item Từ điều kiện KKT hãy tìm nghiệm tối ưu của bài toán.
            
            Các điểm $x, v$ thỏa mãn điều kiện KKT chính là nghiệm $x^*$ của bài toán tối ưu gốc và $v^*$ là nghiệm của bài toán đối ngẫu.

            \begin{equation*}
                \begin{cases} x + A^Tv=0 \\ Ax - y = 0 \end{cases} 
                \Leftrightarrow \begin{cases} x = -A^T v \\ Ax = y \end{cases}
                \Leftrightarrow \begin{cases} x = -A^T v \\ - A A^T v =y \end{cases}
            \end{equation*}

            Ta chứng minh ma trận $AA^T$ khả nghịch. Ta xét hệ phương trình:

            \begin{equation*}
                \begin{aligned}
                    A A^T z = 0 \Rightarrow z^T A A^T z = 0 \Leftrightarrow (A^T z)^T (A^T z)=0 \Leftrightarrow \lVert A^T z \rVert_2^2 = 0
                \end{aligned}
            \end{equation*}

            Do $rank(A)=m$ nên các cột của ma trận $A^T$ (các hàng của ma trận $A$) độc lập tuyến tính vậy để $A^T z = 0 \Leftrightarrow z = 0$. Vậy $AA^Tz=0 \Leftrightarrow z =0$ nên các cột của ma trận $AA^T$ độc lập tuyến tính, ta suy ra $A A^T$ khả nghịch.

            \begin{equation*}
                \Rightarrow \begin{cases} x = -A^T  v \\ v = - (A A^T)^{-1}y \end{cases} \Leftrightarrow \begin{cases} x = A^T  (A A^T)^{-1}y \\ v = - (A A^T)^{-1}y \end{cases}
            \end{equation*}

            Như vậy $x^* = A^T (A A^T)^{-1}y$ là nghiệm của bài toán tối ưu.
            
        \end{enumerate}
    \end{loigiai}

    \newpage
    \printbibliography[title={TÀI LIỆU THAM KHẢO}]
\end{document}