\documentclass[14pt, a4paper]{article}
\usepackage{minitoc}
\usepackage[left=3.00cm, right=2.5cm, top=2.00cm, bottom=2.00cm]{geometry}
\usepackage{amsmath}
\usepackage{amssymb}
\usepackage{amsthm}
\usepackage{mathtools}
\usepackage{graphicx}
\usepackage{algpseudocode}
\usepackage{algorithm}
\usepackage{blindtext}
\usepackage{setspace}
\usepackage[utf8]{inputenc}
\usepackage[utf8]{vietnam}
\usepackage[center]{caption}
\usepackage[shortlabels]{enumitem}
\usepackage{fancyhdr} % header, footer
\usepackage{hyperref} % loại bỏ border với mục lục và công thức
\usepackage[nonumberlist, nopostdot, nogroupskip]{glossaries}
\usepackage{glossary-superragged}
\usepackage{tikz,tkz-tab}
\setglossarystyle{superraggedheaderborder}
\pagestyle{fancy}
%\usepackage[style=numeric,sortcites]{biblatex}
%\addbibresource{ref.bib}
%\usepackage[numbers]{natbib}
\usepackage{indentfirst}
\usepackage[natbib,backend=biber,style=ieee, sorting=ynt]{biblatex}
\bibliography{ref.bib}

\graphicspath{{./figures/}}

%\renewbibmacro*{cite}{%
%  \printtext[bibhyperref]{%
%    \printfield{prefixnumber}%
%    \printfield{labelnumber}%
%    \ifbool{bbx:subentry}%
%      {\printfield{entrysetcount}}%
%   \ifnumequal{\value{citecount}}{\value{citetotal}-1}%
%       {\gdef\multicitedelim{\addspace\bibstring{and}\space}}%
%       {\gdef\multicitedelim{\addcomma\space}}%
%    }%
%}

%\makenoidxglossaries
%
%% Danh mục thuật ngữ
%\newglossaryentry{GMRES}
%{
%	name={GMRES},
%	description={Generalized Minimal Residual Method}
%}
%
%\newglossaryentry{MINRES}
%{
%	name={MINRES},
%	description={Minimal Residual Method}
%}
%
%\hypersetup{
%    colorlinks=false,
%    pdfborder={0 0 0},
%}
%
%\title{Tiểu luận phương pháp số cho đại số tuyến tính}
%
%\author{Nguyễn Chí Thanh}
%
%%\date{24-04-2022}
\fancyhf{}
\rhead{\textbf{Môn học: Tối ưu hóa nâng cao}}
\lhead{\textbf{GVHD: TS. Hoàng Nam Dũng}}
\rfoot{\thepage}
\lfoot{\textbf{Học viên thực hiện: Nguyễn Chí Thanh - Nguyễn Đức Thịnh}}
\renewcommand{\headrulewidth}{0.4pt}
\renewcommand{\footrulewidth}{0.4pt}
%
%\numberwithin{equation}{section}
%\numberwithin{algorithm}{section}
%\numberwithin{figure}{section}
%
%\setlength{\parindent}{0.5cm}
%
%\setcounter{secnumdepth}{3} % Cho phép subsubsection trong report
%\setcounter{tocdepth}{3} % Chèn subsubsection vào bảng mục lục

%\newtheorem{dl}{Định lý}
%\newtheorem{md}{Mệnh đề}
%\newtheorem{bd}{Bổ đề}
%\newtheorem{dn}{Định nghĩa}
%\newtheorem{hq}{Hệ quả}

%\newtheorem{baitap}{Bài tập}
%\newtheorem*{loigiai}{Lời giải}

%\numberwithin{dl}{section}
%\numberwithin{md}{section}
%\numberwithin{bd}{section}
%\numberwithin{dn}{section}
%\numberwithin{hq}{section}

\setlength{\parindent}{0cm}

\newtheorem{dl}{Định lý}
\newtheoremstyle{sltheorem}
{}                % Space above
{}                % Space below
{\normalfont}        % Theorem body font % (default is "\upshape")
{}                % Indent amount
{\bfseries}       % Theorem head font % (default is \mdseries)
{.}               % Punctuation after theorem head % default: no punctuation
{ }               % Space after theorem head
{}                % Theorem head spec
\theoremstyle{sltheorem}
\newtheorem{baitap}{Bài tập}
\newtheoremstyle{soltheorem}
{}                % Space above
{}                % Space below
{\normalfont}        % Theorem body font % (default is "\upshape")
{}                % Indent amount
{\bfseries}       % Theorem head font % (default is \mdseries)
{.}               % Punctuation after theorem head % default: no punctuation
{\newline}               % Space after theorem head
{}                % Theorem head spec
\theoremstyle{soltheorem}
\newtheorem*{loigiai}{Lời giải}

\onehalfspacing

\begin{document}

    \begin{titlepage}

        \newcommand{\HRule}{\rule{\linewidth}{0.5mm}} % Defines a new command for the horizontal lines, change thickness here

        \center % Center everything on the page

        %----------------------------------------------------------------------------------------
        %	HEADING SECTIONS
        %----------------------------------------------------------------------------------------
        \textsc{\LARGE Đại học Quốc Gia Hà Nội}\\[0.5cm]
        \textsc{\LARGE Trường đại học Khoa học tự nhiên}\\[0.5cm] % Name of your university/college
        \textsc{\LARGE Khoa Toán - Cơ - Tin học}\\[0.5cm]

        \includegraphics[scale=0.2]{HUS-logo.jpg}\\[0.5cm]

        \textsc{\Large Chuyên ngành: Khoa học dữ liệu}\\[0.5cm] % Major heading such as course name


        %----------------------------------------------------------------------------------------
        %	TITLE SECTION
        %----------------------------------------------------------------------------------------

        \HRule \\[0.4cm]
        { \huge \bfseries Bài tập môn học}\\[0.4cm] % Title of your document
        \HRule \\[1.5cm]

        \textsc{\Large Môn học: Tối ưu hóa nâng cao}\\[1.5cm] % Minor heading such as course title


        \textsc{\Large Bài tập 4}\\[1.5cm]


        %----------------------------------------------------------------------------------------
        %	AUTHOR SECTION
        %----------------------------------------------------------------------------------------
        \begin{minipage}{0.4\textwidth}
            \begin{flushleft} \Large
            \emph{Giảng viên hướng dẫn:} \\
            TS. Hoàng Nam Dũng % Supervisor's Name
            \end{flushleft}
        \end{minipage}\\[1cm]

        \begin{minipage}{0.4\textwidth}
        \begin{flushleft} \Large
        \emph{Nhóm học viên thực hiện:}\\
        Nguyễn Chí Thanh \\
        MSHV: 21007925 \\ % Your name
        Nguyễn Đức Thịnh \\
        MSHV: 21007923 \\
        Lớp: Khoa học dữ liệu - K4
        \end{flushleft}
        \end{minipage}


        % If you don't want a supervisor, uncomment the two lines below and remove the section above
        %\Large \emph{Author:}\\
        %John \textsc{Smith}\\[3cm] % Your name

        %----------------------------------------------------------------------------------------
        %	DATE SECTION
        %----------------------------------------------------------------------------------------

        % I don't want day because it is English
        % {\large \today}\\[2cm] % Date, change the \today to a set date if you want to be precise

        %----------------------------------------------------------------------------------------
        %	LOGO SECTION
        %----------------------------------------------------------------------------------------

        %\includegraphics{logo/rsz_3logo-khtn.png}\\[1cm] % Include a department/university logo - this will require the graphicx package

        %----------------------------------------------------------------------------------------

        \vfill % Fill the rest of the page with whitespace

    \end{titlepage}

    \nocite{*}

    \newpage

    \begin{baitap}
        Cho $A \in \mathbb{R}^{m\times n}$ và $rank(A)=m$. Xét bài toán tối ưu hóa
        \begin{equation*}
            \begin{aligned}
                &\min \dfrac{1}{2} \lVert x \rVert_2^2 \\ &\text{s.t. }Ax=y
            \end{aligned}
        \end{equation*}
        \begin{enumerate}[wide, labelwidth=!, labelindent=0pt,label=\textbf{\arabic*}.]
            \item Hãy chứng minh rằng bài toán tối ưu trên là một bài toán lồi.
            \item Hãy xây dựng hàm Lagrange của bài toán.
            \item Hãy trình bày điều kiện KKT của bài toán.
            \item Từ điều kiện KKT hãy tìm nghiệm tối ưu của bài toán.
        \end{enumerate}
    \end{baitap}

    \begin{loigiai}

        Cho $A \in \mathbb{R}^{m\times n}$ và $rank(A)=m$. Xét bài toán tối ưu hóa
        \begin{equation*}
            \begin{aligned}
                &\min \dfrac{1}{2} \lVert x \rVert_2^2 \\ &\text{s.t. }Ax=y
            \end{aligned}
        \end{equation*}
        
        \begin{enumerate} [wide, labelwidth=!, labelindent=0pt,label=\textbf{\arabic*}.]
            \item Hãy chứng minh rằng bài toán tối ưu trên là một bài toán lồi.
            
            Ta chứng minh $f(x)=\dfrac{1}{2} \lVert x \rVert_2^2$ là một hàm lồi.
            
            Ta có $dom(f)=\mathbb{R}^n$ là tập lồi.
            \begin{equation*}
                f(x)=\dfrac{1}{2} \lVert x \rVert_2^2=\dfrac{1}{2}x^T x=\dfrac{1}{2}x^T I x
            \end{equation*}

            Ta tính gradient của $f(x)$ theo $x$:

            \begin{equation*}
                \nabla f(x)=\dfrac{1}{2} \Big\lbrack I + I^T \Big\rbrack x=\dfrac{1}{2}2Ix=x
            \end{equation*}

            Ta tính ma trận Hessian của $f(x)$ theo $x$:
            \begin{equation*}
                \nabla^2 f(x)=I
            \end{equation*}

            Ta xét:

            \begin{equation*}
                p^T \nabla^2 f(x) p=p^T I p = p^T p \geq 0 \forall p \in \mathbb{R}^n
            \end{equation*}

            nên:

            \begin{equation*}
                \nabla^2 f(x) \succeq 0 \forall x \in \mathbb{R}^n
            \end{equation*}

            Vì $dom(f)$ là tập lồi và $\nabla^2 f(x) \succeq 0$ nên $f(x)=\dfrac{1}{2} \lVert x \rVert_2^2$ là một hàm lồi.
            Bài toán tối ưu hóa trên có $f(x)$ là hàm lồi, không có ràng buộc bất đẳng thức và ràng buộc đẳng thức $Ax=y$ là hàm affine.
            Vì vậy bài toán tối ưu trên là một bài toán lồi.

            \item Hãy xây dựng hàm Lagrange của bài toán.
            
            Ta xét bài toán tối ưu dạng tổng quát:

            \begin{equation*}
                \begin{aligned}
                    & \min_x f(x) \\
                     \text{s.t } & h_i(x) \leq 0, i = 1, \dots, k \\
                    & l_j(x) = 0, j = 1, \dots, l.
                \end{aligned}
            \end{equation*}
            Hàm Lagrange của bài toán tối ưu được định nghĩa là:

            \begin{equation*}
                L(x, u, v) = f(x) + \sum_{i=1}^k u_i h_i(x) + \sum_{j=1}^l v_j l_j (x)
            \end{equation*}
            với các biến mới $u \in \mathbb{R}^k$ và $v \in \mathbb{R}^l$ và ràng buộc $u \geq 0$.

            Đối với bài toán tối ưu ở đề bài chỉ có ràng buộc đẳng thức mà không có ràng buộc bất đẳng thức nên hàm Lagrange của bài toán tối ưu ở đề bài là:

            \begin{equation*}
                L(x, v)=f(x) + v^T\Big \lbrack Ax - y \Big \rbrack=\dfrac{1}{2}\lVert x \rVert_2^2 + v^T \Big \lbrack Ax - y \Big \rbrack \text{ (do không có ràng buộc bất đẳng thức) } v \in \mathbb{R}^m
            \end{equation*}

            \item Hãy trình bày điều kiện KKT của bài toán.
            
            Ta xét bài toán tối ưu dạng tổng quát:

            \begin{equation*}
                \begin{aligned}
                    & \min_x f(x) \\
                     \text{s.t } & h_i(x) \leq 0, i = 1, \dots, k \\
                    & l_j(x) = 0, j = 1, \dots, l.
                \end{aligned}
            \end{equation*}

            Điều kiện Karush - Kuhn - Tucker (KKT) của bài toán tối ưu này là:

            \begin{equation*}
                \begin{cases}
                    0 \in \partial \Bigg( f(x) + \displaystyle\sum_{i=1}^k u_i h_i(x) + \displaystyle\sum_{j=1}^l v_j l_j(x) \Bigg) \\
                    u_i h_i(x) = 0 \thickspace \forall \thickspace i = 1, \dots, k \\
                    h_i(x) \leq 0, l_j(x) = 0 \thickspace \forall \thickspace i=1, \dots, k, j=1, \dots, l \\
                    u_i \geq 0 \thickspace \forall i = 1, \dots, k
                \end{cases}
            \end{equation*}

            Đối với bài toán tối ưu ở đề bài chỉ có ràng buộc đẳng thức mà không có ràng buộc bất đẳng thức nên điều kiện KKT của bài toán tối ưu ở đề bài là:

            \begin{equation*}
                \begin{aligned}
                &\begin{cases}
                    0 \in \partial \Big ( f(x) + v^T\Big \lbrack Ax - y \Big \rbrack \Big) \\ Ax - y = 0
                \end{cases}\\
                \Leftrightarrow & \begin{cases} 0 \in \partial \Big( \dfrac{1}{2} \lVert x \rVert_2^2 + v^T\Big \lbrack Ax - y \Big \rbrack \Big) \\ Ax - y = 0 \end{cases}
                \end{aligned}
            \end{equation*}

            Ta chứng minh $f(x) + v^T\Big \lbrack Ax - y \Big \rbrack$ là hàm lồi.

            Ta xét hàm số:

            \begin{equation*}
                \begin{aligned}
                    g(x) &= f(x) + v^T\Big \lbrack Ax - y \Big \rbrack \\
                    &=\dfrac{1}{2} \lVert x \rVert_2^2 + v^T\Big \lbrack Ax - y \Big \rbrack
                \end{aligned}
            \end{equation*}

            Ta có $dom(g)= \mathbb{R}^n$ là tập lồi.

            \begin{equation*}
                \begin{aligned}
                    g(x) &= \dfrac{1}{2} \lVert x \rVert_2^2 + v^T\Big \lbrack Ax - y \Big \rbrack \\
                    &=\dfrac{1}{2}x^T x + v^T\Big \lbrack Ax - y \Big \rbrack \\
                    &= \dfrac{1}{2}x^T I x + v^T\Big \lbrack Ax - y \Big \rbrack
                \end{aligned}
            \end{equation*}
            Ta tính gradient của $g(x)$ theo $x$:
            \begin{equation*}
                \nabla g(x) = \dfrac{1}{2} \Big \lbrack I + I^T \Big \rbrack x + A^T v = \dfrac{1}{2}2I x + A^T x = x + A^T v
            \end{equation*}
            Ta tính ma trận Hessian của $g(x)$ theo $x$:

            \begin{equation*}
                \nabla^2 g(x) = I
            \end{equation*}

            Ta xét:

            \begin{equation*}
                p^T \nabla^2 g(x) p = p^T I p = p^T p \geq 0 \thickspace \forall \thickspace p \in \mathbb{R}^n
            \end{equation*}

            nên:

            \begin{equation*}
                \nabla^2 g(x) \succeq 0 \thickspace \forall \thickspace x \in \mathbb{R}^n
            \end{equation*}

            Vì $dom(g)=\mathbb{R}^n$ và $\nabla^2 g(x) \succeq 0$ nên $g(x) = \dfrac{1}{2} \lVert x \rVert_2^2 + v^T\Big \lbrack Ax - y \Big \rbrack$ là hàm lồi.

            Ta nhận thấy $g(x)$ là một hàm khả vi với mọi $x \in \mathbb{R}^n$ và $g(x)$ là một hàm lồi nên ta có thể tính dưới vi phân $\partial g(x)$:

            \begin{equation*}
                \begin{aligned}
                    \partial g(x) &= \lbrace \nabla g(x) \rbrace \\
                    &= \lbrace x + A^T v \rbrace
                \end{aligned}
            \end{equation*}

            Để $0 \in \partial g(x) \Rightarrow \nabla g(x) = 0 \Rightarrow x + A^T v = 0$.
            Ta viết lại điều kiện KKT:

            \begin{equation*}
                \begin{cases} x + A^Tv=0 \\ Ax - y = 0 \end{cases}
            \end{equation*}

            \item Từ điều kiện KKT hãy tìm nghiệm tối ưu của bài toán.
            
            Các điểm $x, v$ thỏa mãn điều kiện KKT chính là nghiệm $x^*$ của bài toán tối ưu gốc và $v^*$ là nghiệm của bài toán đối ngẫu.

            \begin{equation*}
                \begin{cases} x + A^Tv=0 \\ Ax - y = 0 \end{cases} 
                \Leftrightarrow \begin{cases} x = -A^T v \\ Ax = y \end{cases}
                \Leftrightarrow \begin{cases} x = -A^T v \\ - A A^T v =y \end{cases}
            \end{equation*}

            Ta chứng minh ma trận $AA^T$ khả nghịch. Ta xét hệ phương trình:

            \begin{equation*}
                \begin{aligned}
                    A A^T z = 0 \Rightarrow z^T A A^T z = 0 \Leftrightarrow (A^T z)^T (A^T z)=0 \Leftrightarrow \lVert A^T z \rVert_2^2 = 0
                \end{aligned}
            \end{equation*}

            Do $rank(A)=m$ nên các cột của ma trận $A^T$ (các hàng của ma trận $A$) độc lập tuyến tính vậy để $A^T z = 0 \Leftrightarrow z = 0$. Vậy $AA^Tz=0 \Leftrightarrow z =0$ nên các cột của ma trận $AA^T$ độc lập tuyến tính, ta suy ra $A A^T$ khả nghịch.

            \begin{equation*}
                \Rightarrow \begin{cases} x = -A^T  v \\ v = - (A A^T)^{-1}y \end{cases} \Leftrightarrow \begin{cases} x = A^T  (A A^T)^{-1}y \\ v = - (A A^T)^{-1}y \end{cases}
            \end{equation*}

            Như vậy $x^* = A^T (A A^T)^{-1}y$ là nghiệm của bài toán tối ưu.
            
        \end{enumerate}
    \end{loigiai}

    \begin{baitap}
        Hãy giải bài toán tối ưu hóa sau sử dụng ít nhất 2 cách khác nhau

        \begin{equation*}
            \begin{aligned}
                &\min 3 x_1 + 2x_2 \\ &\text{s.t. }x_1^2 - 2 x_1 + x_2^2 = 3
            \end{aligned}
        \end{equation*}
    \end{baitap}

    \begin{loigiai}
        
        Ta đặt $f(x)=f(x_1, x_2) = 3x_1 + 2x_2$

        \textbf{Cách 1:} Sử dụng phương pháp nhân tử Lagrange:

        Ta xét hàm Lagrange của bài toán tối ưu hóa:

        \begin{equation*}
            L(x_1, x_2, \lambda) = 3x_1 + 2x_2 + \lambda ( x_1^2 - 2 x_1 + x_2^2 - 3)
        \end{equation*}
        Điều kiện cần bài toán tối ưu trên đạt cực trị là:

        \begin{equation*}
            \begin{cases} \dfrac{\partial L}{\partial x_1}=0 \\ \dfrac{\partial L}{\partial x_2}=0 \\ \dfrac{\partial L}{\partial \lambda}=0 \end{cases}
        \end{equation*}

        Ta giải hệ phương trình trên để tìm các điểm đạt điều kiện cần trở thành cực trị:

        \begin{equation*}
            \begin{cases} \dfrac{\partial L}{\partial x_1}=0 \\ \dfrac{\partial L}{\partial x_2}=0 \\ \dfrac{\partial L}{\partial \lambda}=0 \end{cases}
            \Leftrightarrow \begin{cases} \dfrac{\partial L}{\partial x_1}=3 + 2\lambda x_1 - 2\lambda=0 \\ \dfrac{\partial L}{\partial x_2}=2 + 2\lambda x_2=0 \\ \dfrac{\partial L}{\partial \lambda}=x_1^2 - 2 x_1 + x_2^2 - 3=0 \end{cases}
        \end{equation*}

        Ta nhận thấy khi $\lambda = 0$ thì phương trình thứ nhất của hệ phương trình trên trở thành $3=0$ (vô lý) và phương trình thứ hai trở thành $2=0$ (vô lý).
        Vì vậy $\lambda = 0$ không thể là nghiệm. Biểu diễn lại $x_1, x_2$ theo $\lambda$ ta được:

        \begin{equation*}
            \begin{cases} x_1 = \dfrac{2\lambda - 3}{2\lambda} \\ x_2 = - \dfrac{1}{\lambda} \\ x_1^2 - 2 x_1 + x_2^2 - 3=0 \end{cases}
        \end{equation*}

        Phương trình thứ ba của hệ phương trình trên trở thành:

        \begin{equation*}
            \begin{aligned}
                &\dfrac{(2\lambda - 3)^2}{(2\lambda)^2}-2 \dfrac{2\lambda -3}{2\lambda} + \dfrac{1}{\lambda^2}-3=0 \\
                \Leftrightarrow &\dfrac{4\lambda^2 - 12\lambda + 9 - 8\lambda^2 + 12\lambda + 4 - 12\lambda^2}{4\lambda^2}=0 \\ \Leftrightarrow &-16\lambda^2 + 13=0 \\
                \Leftrightarrow &\left [\begin{array}{l} \lambda = - \dfrac{\sqrt{13}}{4} \\ \lambda = \dfrac{\sqrt{13}}{4} \end{array} \right.
            \end{aligned}
        \end{equation*}
        \begin{itemize}
            \item Với $\lambda = - \dfrac{\sqrt{13}}{4}$:
            \begin{equation*}
                \Rightarrow \begin{cases} x_1=\dfrac{-2\dfrac{\sqrt{13}}{4} - 3}{-2\dfrac{\sqrt{13}}{4}}=\dfrac{\sqrt{13} + 6}{\sqrt{13}} \\ x_2 = - \dfrac{1}{-\dfrac{\sqrt{13}}{4}}=\dfrac{4}{\sqrt{13}} \\ \lambda = -\dfrac{\sqrt{13}}{4} \end{cases}
            \end{equation*}
            Ta kiểm tra ma trận Hessian $\nabla^2 L= \begin{pmatrix} \dfrac{\partial^2 L}{\partial x_1^2} & \dfrac{\partial^2 L}{\partial x_1 \partial x_2} \\ \dfrac{\partial^2 L}{\partial x_2 \partial x_1} & \dfrac{\partial^2 L}{\partial x_2^2} \end{pmatrix} $ theo $x_1, x_2$ tại các giá trị $(x_1, x_2, \lambda)$ đã được tính ở trên.

            \begin{equation*}
                \begin{aligned}
                    \nabla^2 L \Big( \lambda = - \dfrac{\sqrt{13}}{4} \Big) &= \begin{pmatrix} \dfrac{\partial^2 L}{\partial x_1^2} & \dfrac{\partial^2 L}{\partial x_1 \partial x_2} \\ \dfrac{\partial^2 L}{\partial x_2 \partial x_1} & \dfrac{\partial^2 L}{\partial x_2^2} \end{pmatrix} \\
                    &= \begin{pmatrix} 2\lambda & 0 \\ 0 & 2\lambda \end{pmatrix} \\
                    &= \begin{pmatrix} -\dfrac{\sqrt{13}}{2} & 0 \\ 0 & -\dfrac{\sqrt{13}}{2} \end{pmatrix}
                \end{aligned}
            \end{equation*}

            Ta nhận thấy với $p=\begin{pmatrix} p_1 \\ p_2 \end{pmatrix}$:

            \begin{equation*}
                p^T \nabla^2 L \Big( \lambda = - \dfrac{\sqrt{13}}{4} \Big) p=- \dfrac{\sqrt{13}}{2} (p_1^2 + p_2^2) < 0 \thickspace \forall \thickspace p \in \mathbb{R}^2 \backslash \lbrace 0 \rbrace
            \end{equation*}

            Vì vậy $\nabla^2 L \Big( \lambda = - \dfrac{\sqrt{13}}{4} \Big)$ là ma trận xác định âm. Vậy các nghiệm $(x_1, x_2)$ tương ứng với $\lambda = -\dfrac{\sqrt{13}}{4}$ đã được tính ở trên là một cực đại và không phải là nghiệm của bài toán.
            \item Với $\lambda =  \dfrac{\sqrt{13}}{4}$:
            \begin{equation*}
                \Rightarrow \begin{cases} x_1 = \dfrac{2 \dfrac{\sqrt{13}}{4}-3}{2 \dfrac{\sqrt{13}}{4}}=\dfrac{\sqrt{13}-6}{\sqrt{13}} \\ x_2 = - \dfrac{1}{\dfrac{\sqrt{13}}{4}}=-\dfrac{4}{\sqrt{13}} \\ \lambda = \dfrac{\sqrt{13}}{4} \end{cases}
            \end{equation*}

            Ta kiểm tra ma trận Hessian $\nabla^2 L= \begin{pmatrix} \dfrac{\partial^2 L}{\partial x_1^2} & \dfrac{\partial^2 L}{\partial x_1 \partial x_2} \\ \dfrac{\partial^2 L}{\partial x_2 \partial x_1} & \dfrac{\partial^2 L}{\partial x_2^2} \end{pmatrix} $ theo $x_1, x_2$ tại các giá trị $(x_1, x_2, \lambda)$ đã được tính ở trên.

            \begin{equation*}
                \begin{aligned}
                    \nabla^2 L \Big( \lambda = \dfrac{\sqrt{13}}{4} \Big) &= \begin{pmatrix} \dfrac{\partial^2 L}{\partial x_1^2} & \dfrac{\partial^2 L}{\partial x_1 \partial x_2} \\ \dfrac{\partial^2 L}{\partial x_2 \partial x_1} & \dfrac{\partial^2 L}{\partial x_2^2} \end{pmatrix} \\
                    &= \begin{pmatrix} 2\lambda & 0 \\ 0 & 2\lambda \end{pmatrix} \\
                    &= \begin{pmatrix} \dfrac{\sqrt{13}}{2} & 0 \\ 0 & \dfrac{\sqrt{13}}{2} \end{pmatrix}
                \end{aligned}
            \end{equation*}

            Ta nhận thấy với $p=\begin{pmatrix} p_1 \\ p_2 \end{pmatrix}$:
            
            \begin{equation*}
                p^T \nabla^2 L \Big( \lambda = \dfrac{\sqrt{13}}{4} \Big) p=\dfrac{\sqrt{13}}{2} (p_1^2 + p_2^2) > 0 \thickspace \forall \thickspace p \in \mathbb{R}^2 \backslash \lbrace 0 \rbrace
            \end{equation*}
            Vì vậy $\nabla^2 L \Big( \lambda = \dfrac{\sqrt{13}}{4} \Big)$ là ma trận xác định dương. Vậy các nghiệm $(x_1, x_2)$ tương ứng với $\lambda = \dfrac{\sqrt{13}}{4}$ đã được tính ở trên là một cực tiểu và đây chính là nghiệm của bài toán tối ưu (chỉ có duy nhất một nghiệm cực tiểu).

            Vậy nghiệm của bài toán tối ưu là:

            \begin{equation*}
                \begin{cases} x_1^* = \dfrac{\sqrt{13}-6}{\sqrt{13}} \\ x_2^* = -\dfrac{4}{\sqrt{13}} \end{cases}
            \end{equation*}
            ứng với $\lambda^* = \dfrac{\sqrt{13}}{4}$

            Giá trị nhỏ nhất của hàm $f(x_1, x_2)$ với ràng buộc $x_1^2 - 2x_1 + x_2^2 =3$ là:
            \begin{equation*}
                \begin{aligned}
                f(x_1^*, x_2^*)&=3x_1^* + 2x_2^* \\&= 3 \dfrac{\sqrt{13}-6}{\sqrt{13}} - 2 \dfrac{4}{\sqrt{13}}\\
                &= 3 - 2\sqrt{13}
                \end{aligned}
            \end{equation*} 
        \end{itemize}
        \textbf{Cách 2:} Sử dụng điều kiện KKT:

        Ta có hàm Lagrange tương ứng với bài toán tối ưu:

        \begin{equation*}
            L(x_1, x_2, v) = 3 x_1 + 2 x_2 + v(x_1^2 - 2x_1 + x_2^2 - 3)
        \end{equation*}

        Ta xét điều kiện KKT:

        \begin{equation*}
            \begin{aligned}
            &\begin{cases} 0 \in \partial L(x_1, x_2, v) \\ x_1^2 - 2x_1 + x_2^2 - 3 = 0 \end{cases} \\
            \Rightarrow &\begin{cases} 0 \in \partial \Big ( 3 x_1 + 2 x_2 + v(x_1^2 - 2x_1 + x_2^2 - 3) \Big) \\ x_1^2 - 2x_1 + x_2^2 - 3 = 0  \end{cases}
            \end{aligned}
        \end{equation*}

        Ta chứng minh hàm $L(x_1, x_2, v)$ không là hàm lồi đối với $x_1, x_2$

        Ta xét ma trận Hessian $\nabla^2 L= \begin{pmatrix} \dfrac{\partial^2 L}{\partial x_1^2} & \dfrac{\partial^2 L}{\partial x_1 \partial x_2} \\ \dfrac{\partial^2 L}{\partial x_2 \partial x_1} & \dfrac{\partial^2 L}{\partial x_2^2} \end{pmatrix} $ theo $x_1, x_2$:

        \begin{equation*}
            \begin{aligned}
                \nabla^2 L &= \begin{pmatrix} \dfrac{\partial^2 L}{\partial x_1^2} & \dfrac{\partial^2 L}{\partial x_1 \partial x_2} \\ \dfrac{\partial^2 L}{\partial x_2 \partial x_1} & \dfrac{\partial^2 L}{\partial x_2^2} \end{pmatrix} \\
                &= \begin{pmatrix} 2v & 0 \\ 0 & 2v \end{pmatrix} \\
            \end{aligned}
        \end{equation*}

        Để hàm $L(x_1, x_2, v)$ là một hàm lồi theo $x_1, x_2$ thì $\nabla^2 L \succeq 0 \thickspace \forall \thickspace v \in \mathbb{R}$. 
        Nhưng với $v=-\dfrac{1}{2}$ thì:

        \begin{equation*}
            \nabla^2 L = \begin{pmatrix}
                -1 & 0 \\ 0 & -1
            \end{pmatrix}
        \end{equation*}

        Với $p=\begin{pmatrix}
            p_1 \\ p_2
        \end{pmatrix}$:

        \begin{equation*}
            p^T \nabla^2 L p = -(p_1^2 + p_2^2) < 0 \thickspace \forall \thickspace p \in \mathbb{R}^2 \backslash \lbrace 0 \rbrace
        \end{equation*}

        Tại $v=-\dfrac{1}{2}$, ma trận Hessian $\nabla^2 L \prec 0$. Vì vậy hàm $L(x_1, x_2, v)$ không là một hàm lồi theo $x_1, x_2$

        Ta nhận thấy hàm $L(x_1, x_2, v)$ khả vi nhưng là hàm không lồi vì vậy ta không thể tính dưới vi phân $\partial L(x_1, x_2, v)$ bằng công thức:

        \begin{equation*}
            \partial L = \Bigg\lbrace \begin{pmatrix} \dfrac{\partial L(x_1, x_2, v)}{\partial x_1}, & \dfrac{\partial L(x_1, x_2, v)}{\partial x_2} \end{pmatrix}^T \Bigg\rbrace
        \end{equation*}

        Mặt khác, ta thấy hàm $3x_1 + 2x_2$ vừa là hàm lồi vừa là hàm lõm nên dưới vi phân của hàm này chỉ gồm một phần tử:

        \begin{equation*}
            \partial \Big( 3x_1 + 2 x_2 \Big) = \Big\lbrace \begin{pmatrix}
                3 \\ 2
            \end{pmatrix} \Big\rbrace
        \end{equation*}

        Nhưng với hàm $v(x_1^2 - 2x_1 + x_2^2 - 3)=v \Big \lbrack (x_1 - 1)^2 + x_2^2 - 4 \Big \rbrack$ là hàm lồi hoặc lõm phụ thuộc vào giá trị của $v$


        \textbf{Cách 3:} Sử dụng biến đổi tọa độ:

        Ta biến đổi ràng buộc đẳng thức về dạng:

        \begin{equation*}
            \begin{aligned}
            &x_1^2 - 2x_1 + x_2^2 = 3 \\
            \Leftrightarrow & (x_1 - 1)^2 + x_2^2 = 4
            \end{aligned}
        \end{equation*}

        Ta nhận thấy ràng buộc đẳng thức có dạng là một đường tròn (trên không gian hai chiều) hoặc một mặt trụ (trên không gian ba chiều) tâm tại $(1, 0)$ và bán kính là $2$.

        Ta đặt:
        \begin{equation*}
            \begin{cases}
                x_1 = 1 + 2 \cos \varphi \\
                x_2 = 2 \sin \varphi \\
                0 \leq \varphi \leq 2 \pi
            \end{cases}
        \end{equation*}

        Hàm mục tiêu $f(x)=f(x_1, x_2)$ được viết lại dưới dạng:

        \begin{equation*}
            \begin{aligned}
            f(\varphi) &= 3(1 + 2\cos \varphi) + 2.2 \sin \varphi \\
            & = 3 + 6 \cos \varphi + 4 \sin \varphi
            \end{aligned}
        \end{equation*}

        Bây giờ bài toán tối ưu trở thành:

        \begin{equation*}
            \begin{aligned}
                &\min 3 + 6 \cos \varphi + 4 \sin \varphi \\
                &\text{s.t. } 0 \leq \varphi \leq 2\pi
            \end{aligned}
        \end{equation*}

        Ta xét:

        \begin{equation*}
            \dfrac{d f(\varphi)}{d \varphi}=-6 \sin \varphi + 4 \cos \varphi=0
        \end{equation*}

        Với $\cos \varphi = 0 \Rightarrow \sin \varphi = \pm 1$, phương trình $\dfrac{d f(\varphi)}{d \varphi}=0$ trở thành $\mp 6 = 0$ (vô lý).
        Vậy $\varphi$ tương ứng với $\cos \varphi = 0$ không là nghiệm của phương trình $\dfrac{d f(\varphi)}{d \varphi}=0$.

        Ta chia cả hai vế của phương trình cho $\cos \varphi$:

        \begin{equation*}
            \Rightarrow \tan \varphi = \dfrac{2}{3}
        \end{equation*}

        Sử dụng hệ thức $\dfrac{1}{\cos^2 \varphi}=\tan^2 \varphi + 1$:

        \begin{equation*}
            \Rightarrow \cos^2 \varphi = \dfrac{1}{\tan^2 \varphi + 1}=\dfrac{1}{\dfrac{4}{9} + 1}=\dfrac{9}{13}
        \end{equation*}

        Sử dụng hệ thức $\sin^2 \varphi + \cos^2 \varphi = 1$:

        \begin{equation*}
            \Rightarrow \sin^2 \varphi = 1 - \cos^2 \varphi = 1 - \dfrac{9}{13}=\dfrac{4}{13}
        \end{equation*}

        Từ đây ta suy ra:

        \begin{equation*}
            \left [\begin{array}{l} \begin{cases} \sin \varphi = \dfrac{2}{\sqrt{13}} \\ \cos \varphi = \dfrac{3}{\sqrt{13}} \end{cases} \\ \begin{cases} \sin \varphi = -\dfrac{2}{\sqrt{13}} \\ \cos \varphi = -\dfrac{3}{\sqrt{13}} \end{cases} \end{array} \right.
        \end{equation*}

        \begin{itemize}
            \item Với $\begin{cases} \sin \varphi = \dfrac{2}{\sqrt{13}} \\ \cos \varphi = \dfrac{3}{\sqrt{13}} \end{cases}$:
            
            Ta xét đạo hàm bậc hai:

            \begin{equation*}
                \dfrac{d^2 f(\varphi)}{d \varphi^2} = - 6 \cos \varphi - 4 \sin \varphi
            \end{equation*}

            Tại $\begin{cases} \sin \varphi = \dfrac{2}{\sqrt{13}} \\ \cos \varphi = \dfrac{3}{\sqrt{13}} \end{cases}$:

            \begin{equation*}
                \dfrac{d^2 f(\varphi)}{d \varphi^2} = -6 \dfrac{3}{\sqrt{13}} - 4 \dfrac{2}{\sqrt{13}} = -\dfrac{26}{\sqrt{13}}=-2\sqrt{13} < 0
            \end{equation*}

            Vì vậy, tại $\varphi$ sao cho $\begin{cases} \sin \varphi = \dfrac{2}{\sqrt{13}} \\ \cos \varphi = \dfrac{3}{\sqrt{13}} \end{cases}$, $f(\varphi)$ đạt cực đại nên nghiệm $\varphi$ này không là nghiệm của bài toán.

            \item Với $\begin{cases} \sin \varphi = -\dfrac{2}{\sqrt{13}} \\ \cos \varphi = -\dfrac{3}{\sqrt{13}} \end{cases}$:
            Ta xét đạo hàm cấp 2:
            
            \begin{equation*}
                \dfrac{d^2 f(\varphi)}{d \varphi^2} = - 6 \cos \varphi - 4 \sin \varphi
            \end{equation*}

            Tại $\begin{cases} \sin \varphi = -\dfrac{2}{\sqrt{13}} \\ \cos \varphi = -\dfrac{3}{\sqrt{13}} \end{cases}$:

            \begin{equation*}
                \dfrac{d^2 f(\varphi)}{d \varphi^2} = -6 \dfrac{-3}{\sqrt{13}} - 4 \dfrac{-2}{\sqrt{13}} = \dfrac{26}{\sqrt{13}}=2\sqrt{13} > 0
            \end{equation*}

            Vì vậy, tại $\varphi$ sao cho $\begin{cases} \sin \varphi = -\dfrac{2}{\sqrt{13}} \\ \cos \varphi = -\dfrac{3}{\sqrt{13}} \end{cases}$, $f(\varphi)$ đạt cực tiểu và đây chính là nghiệm của bài toán tối ưu.

            Tại $\varphi^*$ sao cho $\begin{cases} \sin \varphi^* = -\dfrac{2}{\sqrt{13}} \\ \cos \varphi^* = -\dfrac{3}{\sqrt{13}} \end{cases}$, nghiệm của bài toán tối ưu là:

            \begin{equation*}
                \begin{cases}
                    x_1^* = 1 + 2 \cos \varphi^* = 1 + 2 \dfrac{-3}{\sqrt{13}} = \dfrac{\sqrt{13} - 6}{\sqrt{13}} \\
                    x_2^* = 2 \sin \varphi^* = 2 \dfrac{-2}{\sqrt{13}} = -\dfrac{4}{\sqrt{13}}
                \end{cases}
            \end{equation*}
        \end{itemize}
        Giá trị nhỏ nhất của hàm $f(x_1, x_2)$ với ràng buộc $x_1^2 - 2x_1 + x_2^2 =3$ là:
            \begin{equation*}
                \begin{aligned}
                f(x_1^*, x_2^*)&=3x_1^* + 2x_2^* \\&= 3 \dfrac{\sqrt{13}-6}{\sqrt{13}} - 2 \dfrac{4}{\sqrt{13}}\\
                &= 3 - 2\sqrt{13}
                \end{aligned}
            \end{equation*}
        đạt được khi:

        \begin{equation*}
            \begin{cases}
            x_1^* = \dfrac{\sqrt{13} - 6}{\sqrt{13}} \\
            x_2^* = -\dfrac{4}{\sqrt{13}}
            \end{cases}
        \end{equation*}

        \textbf{Cách 4:} Sử dụng phương pháp thế và tam thức bậc hai:

        Ta đặt $z = f(x) = f(x_1, x_2) = 3 x_1 + 2 x_2$
        Bài toán tối ưu ban đầu trở thành:

        \begin{equation*}
            \begin{aligned}
                &\min z \\ &\text{s.t. }x_1^2 - 2 x_1 + x_2^2 = 3
            \end{aligned}
        \end{equation*}

        Thay $x_2 = \dfrac{z - 3 x_1}{2}$ và thế vào ràng buộc đẳng thức ta được:

        \begin{equation*}
            \begin{aligned}
                &x_1^2 - 2x_1 + \dfrac{(z - 3 x_1)^2}{2^2}=3 \\
                \Leftrightarrow & 4 x_1^2 - 8 x_1 + (z - 3 x_1)^2 = 12 \\
                \Leftrightarrow & 4 x_1^2 - 8 x_1 + (z^2 - 6z x_1 + 9 x_1^2) = 12 \\
                \Leftrightarrow & 13 x_1^2 - 2(3z + 4)x_1 + z^2 - 12 = 0
            \end{aligned}
        \end{equation*}

        Ta tìm $z$ sao cho phương trình bậc 2 trên có nghiệm:

        \begin{equation*}
            \begin{aligned}
            &\Delta^{\prime} \geq 0 \\
            \Leftrightarrow & (3z + 4)^2 - 13(z^2 - 12) \geq 0 \\
            \Leftrightarrow & 9z^2 + 24z + 16 - 13 z^2 + 156 \geq 0 \\
            \Leftrightarrow & -4 z^2 + 24z + 172 \geq 0
            \end{aligned}
        \end{equation*}

        Phương trình $-4 z^2 + 24z + 172 = 0$ có hai nghiệm:

        \begin{equation*}
            \left [\begin{array}{l} z = 3 - 2 \sqrt{13} \\ z = 3 + 2 \sqrt{13} \end{array} \right.
        \end{equation*}

        Để $-4 z^2 + 24z + 172 \geq 0 \Rightarrow z \in \lbrack 3 - 2\sqrt{13}; 3 + 2 \sqrt{13} \rbrack$.
        Vậy $\min z = 3 - 2 \sqrt{13}$. Khi $z = 2 - \sqrt{13} \Rightarrow \Delta^{\prime} = 0$.
        Khi $\Delta^{\prime}=0$:

        \begin{equation*}
            \begin{aligned}
                &\Rightarrow x_1^* = \dfrac{3z + 4}{13}=\dfrac{3 (3 - 2\sqrt{13}) + 4}{13}=\dfrac{13 - 6\sqrt{13}}{13}=\dfrac{\sqrt{13} - 6}{\sqrt{13}} \\
                &\Rightarrow x_2^* = \dfrac{z - 3x_1^*}{2}=\dfrac{3 - 2\sqrt{13} - 3 \dfrac{\sqrt{13} - 6}{\sqrt{13}}}{2}=\dfrac{-26 + 18}{2\sqrt{13}}=-\dfrac{4}{\sqrt{13}}
            \end{aligned}
        \end{equation*}

        Giá trị nhỏ nhất của hàm $f(x_1, x_2)$ với ràng buộc $x_1^2 - 2x_1 + x_2^2 = 3$ là:

        \begin{equation*}
            f(x_1^*, x_2^*) = \min z = 3 - 2\sqrt{13}
        \end{equation*}

        đạt được khi:
        \begin{equation*}
            \begin{cases} 
                x_1^* = \dfrac{\sqrt{13} - 6}{\sqrt{13}} \\ 
                x_2^* = - \dfrac{4}{\sqrt{13}} 
            \end{cases}
        \end{equation*}

        \textbf{Cách 5:} Sử dụng phương pháp hình học:

        Ta biến đổi ràng buộc đẳng thức về dạng:

        \begin{equation*}
            \begin{aligned}
            &x_1^2 - 2x_1 + x_2^2 = 3 \\
            \Leftrightarrow & (x_1 - 1)^2 + x_2^2 = 4
            \end{aligned}
        \end{equation*}

        Trong mặt phẳng $Ox_1 x_2$, đây là phương trình đường tròn, đặt tâm tại $I(1, 0)$ và bán kính $R=2$:

        Đường thẳng $3x_1 + 2x_2 = 0$, chia mặt phẳng $O x_1 x_2$ thành 2 miền:

        \begin{equation*}
            \mathcal{D}_I = \lbrace (x_1, x_2) \in \mathbb{R}^2: 3x_1 + 2x_2 \geq 0 \rbrace
        \end{equation*}

        \begin{equation*}
            \mathcal{D}_{II} = \lbrace (x_1, x_2) \in \mathbb{R}^2: 3x_1 + 2x_2 < 0 \rbrace
        \end{equation*}

        Nghiệm tối ưu là điểm $H$ là giao giữa đường thẳng đi qua tâm $I$ vuông góc với đường thẳng $3x_1 + 2x_2 = 0$ và đường tròn $(x_1 - 1)^2 + x_2^2 = 4$ và $H \in \mathcal{D}_{II}$.

        Gọi $H(a, b)$:

        \begin{equation*}
            \Rightarrow \overrightarrow{IH}=(a - 1, b)
        \end{equation*}

        Điều kiện để điểm $H$ là nghiệm tối ưu:

        \begin{equation*}
            \begin{aligned}
            &\begin{cases}
                \overrightarrow{IH} = k(3, 2) \\
                \lVert IH \rVert = R = 2 \\
                H \in \mathcal{D}_{II}
            \end{cases} \\
            \Leftrightarrow & \begin{cases}
                \overrightarrow{IH} = k(3, 2) \\
                \lVert IH \rVert = R = 2 \\
                3a + 2b < 0
            \end{cases} \\
            \Leftrightarrow & \begin{cases}
                (a - 1, b) = k(3, 2) \\
                \sqrt{(a - 1)^2 + b^2} = 2 \\
                3a + 2b < 0
            \end{cases} \\
            \Leftrightarrow & \begin{cases}
                a - 1 = 3k \\
                b = 2k \\
                (a - 1)^2 + b^2 = 4 \\
                3a + 2b < 0
            \end{cases} \\
            \Leftrightarrow & \begin{cases}
                a - 1 = 3k \\
                b = 2k \\
                (3k)^2 + (2k)^2 = 4 \\
                3a + 2b < 0
            \end{cases} \\
            \Leftrightarrow & \begin{cases}
                a - 1 = 3k \\
                b = 2k \\
                k^2 = \dfrac{4}{13} \\
                3a + 2b < 0
            \end{cases} \\
            \Leftrightarrow & \begin{cases}
                a = \dfrac{\sqrt{13} - 6}{\sqrt{13}} \\
                b = -\dfrac{4}{\sqrt{13}} \\ 
                k = - \dfrac{2}{\sqrt{13}} (k=\dfrac{2}{\sqrt{13}} \text{ không thỏa mãn điều kiện } 3a + 2b < 0)
            \end{cases}
            \end{aligned}
        \end{equation*}

        Giá trị nhỏ nhất của hàm $f(x_1, x_2)$ với ràng buộc $x_1^2 - 2x_1 + x_2^2 = 3$ là:

        \begin{equation*}
            f(x_1^*, x_2^*) = 3 a + 2b = 3 \dfrac{\sqrt{13} - 6}{\sqrt{13}} - 2 \dfrac{4}{\sqrt{13}}  = 3 - 2\sqrt{13}
        \end{equation*}

        đạt được khi:
        \begin{equation*}
            \begin{cases} 
                x_1^* = a = \dfrac{\sqrt{13} - 6}{\sqrt{13}} \\ 
                x_2^* = a = - \dfrac{4}{\sqrt{13}} 
            \end{cases}
        \end{equation*}

        \textbf{Cách 6:} Sử dụng bất đẳng thức Cauchy - Schwarz:

        Ta biến đổi ràng buộc đẳng thức về dạng:

        \begin{equation*}
            \begin{aligned}
            &x_1^2 - 2x_1 + x_2^2 = 3 \\
            \Leftrightarrow & (x_1 - 1)^2 + x_2^2 = 4
            \end{aligned}
        \end{equation*}

        Sử dụng bất đẳng thức Cauchy - Schwarz:

        \begin{equation*}
            \begin{aligned}
            &\Big ( 3(x_1 - 1) + 2x_2 \Big )^2 \leq (3^2 + 2^2)\Big ( (x_1 - 1)^2 + x_2^2 \Big) = 13\times4 = 52 \\
            \Leftrightarrow & - 2\sqrt{13} \leq 3(x_1 - 1) + 2x_2 \leq 2 \sqrt{13} \\
            \Leftrightarrow & - 2 \sqrt{13} \leq 3 x_1 + 2 x_2 - 3 \leq 2 \sqrt{13} \\
            \Leftrightarrow & - 2 \sqrt{13} \leq f(x_1, x_2) - 3 \leq 2 \sqrt{13} \\
            \Leftrightarrow & 3 - 2 \sqrt{13} \leq f(x_1, x_2) \leq 3 + 2 \sqrt{13}
            \end{aligned}
        \end{equation*}

        Dấu "=" xảy ra khi và chỉ khi:

        \begin{equation*}
            \begin{aligned}
                &\begin{cases}
                    \dfrac{x_1 - 1}{3} = \dfrac{x_2}{2} \\
                    (x_1 - 1)^2 + x_2^2 = 4
                \end{cases} \\
                \Leftrightarrow & \begin{cases}
                    x_2 = \dfrac{2(x_1 - 1)}{3} \\
                    (x_1 - 1)^2 + x_2^2 = 4
                \end{cases} \\
                \Leftrightarrow & \begin{cases}
                    x_2 = \dfrac{2(x_1 - 1)}{3} \\
                    (x_1 - 1)^2 + \dfrac{4}{9}(x_1 - 1)^2 = 4
                \end{cases} \\
                \Leftrightarrow & \begin{cases}
                    x_2 = \dfrac{2(x_1 - 1)}{3} \\
                    \dfrac{13}{9}(x_1 - 1)^2 = 4
                \end{cases} \\
                \Leftrightarrow & \left [\begin{array}{l} \begin{cases} x_1 = \dfrac{\sqrt{13} - 6}{\sqrt{13}} \\ x_2 = -\dfrac{4}{\sqrt{13}} \end{cases} \\ \begin{cases} x_1 = \dfrac{\sqrt{13} + 6}{\sqrt{13}} \\ x_2 = \dfrac{4}{\sqrt{13}} \end{cases} \end{array} \right.
            \end{aligned}
        \end{equation*}

        Ta nhận thấy $f(x_1=\dfrac{\sqrt{13} - 6}{\sqrt{13}}, x_2 = -\dfrac{4}{\sqrt{13}})=3 - 2\sqrt{13}$

        Giá trị nhỏ nhất của hàm $f(x_1, x_2)$ với ràng buộc $x_1^2 - 2x_1 + x_2^2 = 3$ là:

        \begin{equation*}
            f(x_1^*, x_2^*) = 3 - 2\sqrt{13}
        \end{equation*}

        đạt được khi:
        \begin{equation*}
            \begin{cases} 
                x_1^* = \dfrac{\sqrt{13} - 6}{\sqrt{13}} \\ 
                x_2^* = - \dfrac{4}{\sqrt{13}} 
            \end{cases}
        \end{equation*}

        \textbf{Cách 7:} Biến đổi và đánh giá trực tiếp hàm mục tiêu:

        Ta đặt $z = f(x) = f(x_1, x_2) = 3 x_1 + 2 x_2$

        Ta xét:

        \begin{equation*}
            \begin{aligned}
            \dfrac{4}{\sqrt{13}} z &= \dfrac{4}{\sqrt{13}} (3 x_1 + 2 x_2 ) + 3 - 3 \\
                                & = \dfrac{4}{\sqrt{13}} (3 x_1 + 2 x_2 ) + (x_1^2 - 2x_1 + x_2^2) - 3 \\
                                & = \dfrac{12}{\sqrt{13}} x_1 + \dfrac{8}{\sqrt{13}} x_2 + (x_1^2 - 2x_1 + x_2^2) - 3 \\
                                & = \Big ( x_1^2 + \dfrac{12}{\sqrt{13}} x_1 - 2 x_1 + \dfrac{36}{13} + 1 \Big) + \Big ( x_2^2 + \dfrac{8}{\sqrt{13}}x_2 + \dfrac{16}{13} \Big) + \dfrac{12}{\sqrt{13}} - 8 \\
                                & = \Big ( x_1 + \dfrac{6}{\sqrt{13}} - 1 \Big)^2 + \Big( x_2 + \dfrac{4}{\sqrt{13}} \Big)^2 + \dfrac{12}{\sqrt{13}} - 8 \geq \dfrac{12}{\sqrt{13}} - 8 \\
                        \Leftrightarrow z &\geq 3 - 2\sqrt{13}
            \end{aligned}
        \end{equation*}
    \end{loigiai}

    Dấu "=" xảy ra khi và chỉ khi:

    \begin{equation}
        \begin{aligned}
        &\begin{cases}
            x_1 + \dfrac{6}{\sqrt{13}} - 1 = 0 \\ 
            x_2 + \dfrac{4}{\sqrt{13}} = 0 
        \end{cases} \\
        \Leftrightarrow & \begin{cases}
            x_1 = \dfrac{\sqrt{13} - 6}{\sqrt{13}} \\ 
            x_2 = - \dfrac{4}{\sqrt{13}} 
        \end{cases}
        \end{aligned}
    \end{equation}

    Giá trị nhỏ nhất của hàm $f(x_1, x_2)$ với ràng buộc $x_1^2 - 2x_1 + x_2^2 = 3$ là:

    \begin{equation*}
        f(x_1^*, x_2^*) = 3 - 2\sqrt{13}
    \end{equation*}

    đạt được khi:
    \begin{equation*}
        \begin{cases} 
            x_1^* = \dfrac{\sqrt{13} - 6}{\sqrt{13}} \\ 
            x_2^* = - \dfrac{4}{\sqrt{13}} 
        \end{cases}
    \end{equation*}

    \textbf{Cách 8:} Sử dụng phương pháp thế và khảo sát hàm mục tiêu:

    Từ ràng buộc đẳng thức $x_1^2 - 2x_1 + x_2^2 = 3$, ta thu được $x_2$:

    \begin{equation*}
        \begin{aligned}
            &x_2 = \pm \sqrt{3 + 2x_1 - x_1^2} \\
            \text{với }& -1 \leq x_1 \leq 3 \text{ Điều kiện để biểu thức trong căn có nghĩa}
        \end{aligned}
    \end{equation*}

    Hàm mục tiêu $f(x_1, x_2)$ trở thành:

    \begin{equation*}
        f(x_1, x_2) = \left [\begin{array}{l} 3x_1 + 2\sqrt{3 + 2x_1 - x_1^2} \\ 3x_1 - 2\sqrt{3 + 2x_1 - x_1^2} \end{array} \right.
    \end{equation*}

    với $-1 \leq x_1 \leq 3$

    Ta nhận thấy $3x_1 + 2\sqrt{3 + 2x_1 - x_1^2} \geq 3x_1 - 2\sqrt{3 + 2x_1 - x_1^2} \thickspace \forall \thickspace -1 \leq x_1 \leq x_3$,
    nên $\min f(x_1, x_2) = \min 3x_1 - 2\sqrt{3 + 2x_1 - x_1^2} $ với $-1 \leq x_1 \leq 3$

    Đặt $g(x_1) = 3x_1 - 2\sqrt{3 + 2x_1 - x_1^2}$ với $-1 \leq x_1 \leq 3$

    Ta xét:

    \begin{equation*}
        g^{\prime}(x_1) = 3 + \dfrac{2 x_1 - 2}{\sqrt{3 + 2x_1 - x_1^2}}
    \end{equation*}

    Ta giải phương trình $g^{\prime}(x_1) = 0$:

    \begin{equation*}
        \begin{aligned}
            &g^{\prime}(x_1) = 0 \\
            \Leftrightarrow &\dfrac{2 x_1 - 2}{\sqrt{3 + 2x_1 - x_1^2}} = -3 \\
            \Leftrightarrow &\begin{cases}
                x_1 \leq 1 \\
                -1 \leq x_1 \leq 3 \\
                2(x_1 - 1) = -3\sqrt{3 + 2x_1 - x_1^2}
            \end{cases} \\
            \Leftrightarrow & \begin{cases}
                -1 \leq x_1 \leq 1 \\
                4(x_1 - 1)^2 = 9(3 + 2x_1 - x_1^2)
            \end{cases} \\
            \Leftrightarrow & \begin{cases}
                -1 \leq x_1 \leq 1 \\
                13x_1^2 - 26 x_1 - 23 = 0
            \end{cases}
            \Leftrightarrow & \begin{cases}
                -1 \leq x_1 \leq 1 \\
                \left [ \begin{array}{l} x_1 = \dfrac{6 - \sqrt{13}}{\sqrt{13}} \\ x_1 = \dfrac{6 + \sqrt{13}}{\sqrt{13}} \text{ (Không thỏa mãn điều kiện } -1 \leq x_1 \leq 1 \text{ )} \end{array}\right.
            \end{cases}
        \end{aligned}
    \end{equation*}

    Ta lập bảng biến thiên của hàm số $g(x_1)$ với $-1 \leq x_1 \leq 3$:

    \begin{tikzpicture}
        \tkzTabInit
        [lgt=3,espcl=5] % tùy chọn
        {$x_1$/1.2, $g^{\prime}(x_1)$/1, $g(x_1)$/2.5} % cột đầu tiên
        {$-1$, $\dfrac{6 - \sqrt{13}}{\sqrt{13}}$, $3$} % hàng 1 cột 2
        \tkzTabLine{,-,0,+,} % hàng 2 cột 2
        \tkzTabVar{+/ $-3$, -/ $3 - 2\sqrt{13}$ , +/ $9$} % hàng 3 cột 2
    \end{tikzpicture}

    Vậy $g(x_1)$ với $-1 \leq x_1 \leq 3$ đạt giá trị nhỏ nhất tại $x_1 = \dfrac{6 - \sqrt{13}}{\sqrt{13}}$.
    Ta tính $x_2$:

    \begin{equation}
        x_2 = - \sqrt{3 + 2x_1 - x_1^2}= \sqrt{3 + 2 \dfrac{6 - \sqrt{13}}{\sqrt{13}} - \Big( \dfrac{6 - \sqrt{13}}{\sqrt{13}} \Big)^2} = -\dfrac{4}{\sqrt{13}}
    \end{equation}

    Giá trị nhỏ nhất của hàm $f(x_1, x_2)$ với ràng buộc $x_1^2 - 2x_1 + x_2^2 = 3$ là:

    \begin{equation*}
        f(x_1^*, x_2^*) = g(x_1^{*}) = 3 - 2\sqrt{13}
    \end{equation*}

    đạt được khi:
    \begin{equation*}
        \begin{cases} 
            x_1^* = \dfrac{\sqrt{13} - 6}{\sqrt{13}} \\ 
            x_2^* = - \dfrac{4}{\sqrt{13}} 
        \end{cases}
    \end{equation*}

    \newpage
    \printbibliography[title={TÀI LIỆU THAM KHẢO}]
\end{document}